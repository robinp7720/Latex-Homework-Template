\documentclass[a4paper, 11pt]{article}

% Layout
\usepackage[a4paper, left=3cm, right=3cm, top=2cm, bottom=3cm]{geometry} % kleinere Ränder
\usepackage{parskip}

\usepackage{subfiles}

% Umlaute in der Datei erlauben, auf Deutsch umstellen
\usepackage[T1]{fontenc}
\usepackage{lmodern}
\usepackage[utf8]{inputenc}
\usepackage[english, ngerman]{babel} % Abgaben auf DEUTSCH
%\usepackage[english]{babel} % for submissions in ENGLISH

% Mathesymbole und Ähnliches
\usepackage{amsmath}
\usepackage{mathtools}
\usepackage{amssymb}
\usepackage{microtype}
\usepackage{stmaryrd}

% Grafiken und PDFs einfügen
\usepackage{graphicx}
\usepackage{pdfpages}

% Abbildungen
\usepackage{tikz}
\usetikzlibrary{arrows, calc}

% Reelle, Natürliche, Ganze, Rationale Zahlen
\newcommand{\R}{\ensuremath{\mathbb{R}}}
\newcommand{\N}{\ensuremath{\mathbb{N}}}
\newcommand{\Z}{\ensuremath{\mathbb{Z}}}
\newcommand{\Q}{\ensuremath{\mathbb{Q}}}

% Fraktur für Strukturen
\newcommand{\A}{\ensuremath{\mathfrak A}}
\newcommand{\B}{\ensuremath{\mathfrak B}}
\newcommand{\I}{\ensuremath{\mathfrak I}}

% Makros für logische Operatoren
\newcommand{\xor}{\ensuremath{\oplus}} % exklusives oder
\newcommand{\impl}{\ensuremath{\rightarrow}} % logische Implikation

% Meistens ist \varphi schöner als \phi, genauso bei \theta
\renewcommand{\phi}{\varphi}
\renewcommand{\theta}{\vartheta}

% Aufzählungen anpassen (alternativ: \arabic, \alph)
\renewcommand{\labelenumi}{(\roman{enumi})}

\begin{document}

% OPTIONAL: Kopfzeile / header
Subject~SoSe~2022
\hfill
{\Large
    Übungsblatt 0 % <--- Blattnummer / sheet number
}
\hfill
\today

% Bitte prüfen Sie Ihre MOODLE-ABGABEGRUPPE, damit die Abgabe Ihnen richtig zugeordnet wird.
% Sie KÖNNEN hier Ihre Namen und Matrikelnummern für sich angeben, aber die Angaben werden IGNORIERT.

% Please check your MOODLE SUBMISSION GROUP in order to ensure that your submission is attributed to you.
% You MAY provide your names and matriculation numbers here for your own use, but they will be IGNORED.

\hrule

\section*{Aufgabe 1}
\subfile{Aufgaben/robin_aufgabe1.tex}

\end{document}
